\documentclass[]{article}

%opening
\title{EE131A Class Project}
\author{Alon Krauthammer}
\date{November 28, 2020}
\usepackage{amsmath,amsfonts,amssymb,amsthm}
\usepackage{braket}
\usepackage{tikz}
\usepackage{breqn}
\usetikzlibrary{shapes,backgrounds}
\usepackage{graphicx}
\usepackage[margin=1.0in]{geometry}
\usepackage{bbold}
\graphicspath{ {./images/} }
\usepackage{listings}
\usepackage{color}

\definecolor{dkgreen}{rgb}{0,0.6,0}
\definecolor{gray}{rgb}{0.5,0.5,0.5}
\definecolor{mauve}{rgb}{0.58,0,0.82}

\lstset{frame=tb,
	language=Python,
	aboveskip=3mm,
	belowskip=3mm,
	showstringspaces=false,
	columns=flexible,
	basicstyle={\small\ttfamily},
	numbers=none,
	numberstyle=\tiny\color{gray},
	keywordstyle=\color{blue},
	commentstyle=\color{dkgreen},
	stringstyle=\color{mauve},
	breaklines=true,
	breakatwhitespace=true,
	tabsize=3
}

\begin{document}

\maketitle

All the code written for this project is downloadable as python notebooks from my github:\\

https://github.com/krauthammera/ee131a_finalproj

\begin{enumerate}
	\item Tossing  a  fair  and  unfair  die. Suppose  you  have  a  4-sided  die,  with  sides numbered 1, 2, 3, and 4.
	\begin{enumerate}
		\item Write a MATLAB program to simulate the tossing of a 4-sided fair die, for t=10, 50, 100, 500 and 1000  tosses. Based on the simulation, what is the  estimated probability of obtaining an odd number?\\
		
		\begin{lstlisting}
			import numpy as np
			import pandas as pd
			import matplotlib.pyplot as plt
			
			#Tossing a fair and unfair die, 4 sides
			die = np.arange(1,5)
			probs_unbiased = [(1/4)]*4
			probs_biased = [(1/6), (1/3)]*2
			n_throws = [10, 50, 500, 1000]
			
			odd_probs = []
			
			for t in n_throws:
			    #throw the fair die t times
			    outcomes = np.random.choice(die, t, p=probs_unbiased) 
			
			    unique, counts = np.unique(outcomes, return_counts=True)
			    counts_dict = dict(zip(unique, counts))
			
			    #check if all values are in unique, if not, add zero count:
			    for value in die:
			        if value in unique:
			            continue
		        	else:
			            counts_dict[value] = 0
			
			    counts = counts_dict.values()
			    counts_series = pd.Series(counts)
			
			    plt.figure(figsize=(12, 8))
			    ax = counts_series.plot(kind='bar')
			    ax.set_title('Fair Die Roll Frequency, t={}'.format(t))
			    ax.set_xlabel('Value Rolled')
			    ax.set_ylabel('Frequency')
			    ax.set_xticklabels(die)
			
			    #find P(odd)
			    total_odds = counts_dict[1] + counts_dict[3]
			    sim_prob_odd = total_odds/t
			    odd_probs.append(sim_prob_odd)
			
			print("Simulation probability of rolling an odd value converges to {}".format(odd_probs[-1]))
			
			
			plt.figure(figsize=(12, 8))
			plt.plot(np.arange(1, len(n_throws) + 1), odd_probs, '-o')
			plt.ylim(0, 1)
			plt.axhline(y=0.5, color='r', linestyle='dashed', alpha = 0.4)
			plt.title("Convergence of Probability of Rolling an Odd Value - Fair Die")
			
		\end{lstlisting}
		
	    The plot resulting from this simulation is below:\\
		
		\includegraphics[scale=0.5]{fairdie_p_odd}
		
		As can be seen in the plot, the simulation converged to a proportion of odd results of 0.496.\\
		
		\item Suppose \textit{X} is a random variable denoting the outcome of a die toss. Based on the mathematical analysis, what is the probability that \textit{X} has odd value?\\
		
		The sample space of this experiment is 
		
		\[
		S = \{1, 2, 3, 4\}
		\]
		
		Let event \textit{O} be the event that the outcome of the die roll is an odd number. Then $O = {1, 3}$. Because this is a fair die, each of the 4 outcomes is equiprobable, so $P[O] = \frac{1}{4} + \frac{1}{4} = \frac{1}{2}$.\\
		
		\item Refer back to part (a).  Does it agree with the theoretical result in (b)?\\
		
		Yes, this agrees with the simulation, because the simulated value is very close to the theoretical result.\\
		
		\item Repeat parts (a), (b), and (c) if even sides are twice as likely as odd sides, and both even sides are equiprobable and both odd sides are equiprobable.\\
		
		Whether the die is fair or unfair does not change the sample space. But now, the probability of the outcomes ${2, 4}$ are $\frac{1}{3}$ each, and the probability of the outcomes ${1, 3}$ are $\frac{1}{6}$ each. As a result, $P[O] = \frac{1}{6} + \frac{1}{6} = \frac{1}{3}$.\\
		
		The solution to this part is the same as that of part (a), but I used the \texttt{probs\_biased} array as an input to the \texttt{np.random.choice()} function, as seen below:
		
		\begin{lstlisting}
			probs_biased = [(1/6), (1/3)]*2
			
			for t in n_throws:
				#throw the unfair die t times
				outcomes = np.random.choice(die, t, p=probs_biased) 
		\end{lstlisting}
	
		The below plot shows the proportion of rolls resulting in an odd value. For 1000 throws, this value converged to 0.335 in my simulation, which is very close to the theoretical value of $\frac{1}{3}$.\\
	
	    \includegraphics[scale=0.5]{unfair_p_odd}
	    
		
	\end{enumerate}

\pagebreak

\item Consider the binary symmetric channel (BSC) and repetition code described in Lecture 6. The BSC has probability p to flip a bit sent through this channel, with $ 0 < p < 0.5 $.  To transmit information reliably through this channel, we use the N−repetition code  which  repeats  the  bit  we  want  to  transmit \textit{N} times  (known  as  encoding), and then transmits this N-bit sequence through the BSC. Upon receiving this sequence, the receiver uses the majority rule to decide the encoded bit (this is decoding).  For example, for $ N = 5 $,  suppose  we  want  to  transmit  a  0.   We  first  encode  it  into  the sequence  \textit{00000},  using  the  5-repetition  code.   Suppose  some  bits  got  flipped  during transmission over this BSC, so that the receiver observes the sequence \textit{00011}.  In this example, using majority vote, the receiver would determine that the encoded bit for transmission was a 0.  In this question, assume that the bits encoded for transmission are equiprobable.

	\begin{enumerate}
		\item Write a program to simulate the BSC. Also write programs for encoding (repetition code) and decoding (majority rule).  We choose $ N = 1,3,5,7 $ for this problem.  For a given \textit{N}, simulate 20000 instances of sending a bit through the BSC using repetition code and record the fraction of error,  i.e.,  empirical error probability,  for $ p={0.01, 0.02, \ldots, 0.1} $.   Plot  the  empirical  error  probability against \textit{p} for each \textit{N}.  You may want to use a semi-log for the y axis.
		
		First, I imported some packages and defined some constants:
		
		\begin{lstlisting}
			import numpy as np
			import pandas as pd
			import matplotlib.pyplot as plt
			
			from bitarray import bitarray
			from collections import Counter
			from scipy.special import comb
			
			import seaborn as sns
			sns.set(color_codes=True)
			# settings for seaborn plot sizes
			sns.set(rc={'figure.figsize':(20,10)})
			
			p_errs = np.arange(0.01, 0.11, 0.01)
			p_errs_2 = np.arange(0.1, 1.0, 0.05)
			N_vals = [1, 3, 5, 7]
			n_trials = 20000
		\end{lstlisting}
	
		Next, I defined the functions required to perform this simulation:
		
		\begin{lstlisting}
			def get_bit():
				return np.random.choice(2)
			
			def encode(bit, N):
				#bit will be either 0 or 1
				encoded = bitarray([bit]*N)
				return encoded
			
			def channelize(tx_vector, p_err):
				tx_channelized = bitarray(len(tx_vector))
				for ix, bit in enumerate(tx_vector):
					bit = bit
					value = np.random.random_sample()
					if value < p_err:
						#flip bit
						tx_channelized[ix] = not bit
					else:
						tx_channelized[ix] = bit
				return tx_channelized
		\end{lstlisting}
	
		Finally, I carried out the BSC simulation as described. I've left in some debugging print statements as comments, which can be uncommented in order to see the structure of the program:
		
		\begin{lstlisting}
			df_cols = ['N', 'Pe', 'Original', 'Channelized', 'Decision', 'Correct']
			
			error_count = 0
			rows = []
			
			for N in N_vals:
				print("Computing for N = {}".format(N))
				for p_err in p_errs:
					print("    Computing for Pe = {}".format(p_err))
					for i in range(0, n_trials):
						#print("        trial #{}".format(i))
						bit = get_bit()
						rep_coded = encode(bit, N)
						#print("            Encoded bitstream: {}".format(rep_coded))
						to_tx = channelize(rep_coded, p_err)
						#print("            After applying channel: {}".format(to_tx))
						num0s = Counter(to_tx)[False]
						num1s = Counter(to_tx)[True]
						#print("            0s in Rx: {}".format(num0s))
						#print("            1s in Rx: {}".format(num1s))
				
						if (num0s > num1s):
							rx = 0
						else:
							rx = 1
				
						if (rx != bit):
							#print("ERROR")
							error_count += 1
							row = [N, p_err, rep_coded, to_tx, rx, False]
							#errs_for_given_p += 1
						else:
							row = [N, p_err, rep_coded, to_tx, rx, True]
				
						rows.append(row)
			
			
			print("Total errors in experiment: {}".format(error_count))
			
			#place in a pandas DF in order to calculate p_error for each run easily
			df = pd.DataFrame(rows, columns=df_cols)
			
			experiment_errors = []
			
			#grab a separate pandas df for each N value
			for N in N_vals:
				df_N = df[df["N"]==N]
				
				#initialize the errors list
				errors_for_N = []
				
				#for each p_err, calculate the p_err for the given N
				for p_err in p_errs:
					#get the total set of trials for a specific N and Pe
					df_N_perr = df_N[df_N["Pe"]==p_err]
			
					n_errs = df_N_perr[df_N_perr["Correct"] == False].shape[0]
					p_error = n_errs/df_N_perr.shape[0]
					errors_for_N.append(p_error)
			
				experiment_errors.append(errors_for_N)
			
			#generate another DF for the final dataset, p_error for each value of Pe and N
			exp_errs_df = pd.DataFrame(np.array(experiment_errors).T.tolist(), columns=[str(N) for N in N_vals])
			exp_errs_df["P"] = p_errs
			
			#finally, plot this thing
			plt.figure()
			
			for N in N_vals:
				sns.lineplot(data=exp_errs_df, x="P", y=str(N), label="N = {}".format(N))
			
			plt.xlabel("Bit Error Probability")
			plt.ylabel("Rx Decode Error Probability")
			plt.title("Rx Decode Error Probability for Given Individual Bit Error Probability, BSC")
			plt.yscale('log')
			plt.legend()
			
			
		\end{lstlisting}
	
		This is the plot generated by the above program:\\
		\includegraphics[scale=0.35]{bsc_sim_probs}
		
		For $N=5$ and $N=7$, the $p_{error}$ for small values of $p$ results in zero RX decode errors for this simulation, which is why there is a discontinuity in those plot lines.
		
		
		\item  Find the theoretical error probability of sending a bit through BSC using repeti-tion code, parameterized by N and p. For $ N = 1,3,5, 7 $ and $ p={0.01,0.02,\ldots,0.1} $, generate plots similar to (a) using your theoretical results.\\
		
		For a given N, a majority of the bits must be erroneous for there to be a decode error. As a result, we must calculate the number of ways we can have $floor(\frac{N}{2})+1, \ldots, N$ bit errors and sum these terms. This can be expressed as:
		
		\[
		\sum_{k=floor(\frac{N}{2})+1}^{N}{N \choose k}P_e^k(1-P_e)^{N-k}
		\]
		
		Below is the code which implements the above expression:
		
		\begin{lstlisting}
			experiment = {}
			
			for N in N_vals:

				p_errors = []
			
				for Pe in p_errs:
			
					#print("    {}".format(Pe))
					summation = 0
			
					for k in range(int(N/2)+1, N+1):
						#print("        {}".format(k))
						term = comb(N, k)*(Pe**k)*((1-Pe)**(N-k))
						summation += term
			
					p_errors.append(summation)
			
				experiment[str(N)] = p_errors
				
				#convert dictionary into dataframe
				theoretical_df = pd.DataFrame(experiment)
				theoretical_df["P"] = p_errs
				
				
		\end{lstlisting}
	
		Here is the code used to plot the generated theoretical values:\newline
		
		\begin{lstlisting}
			plt.figure()
			
			for N in N_vals:
				sns.lineplot(data=theoretical_df, x="P", y=str(N), label="N = {}".format(N))
			
			plt.xlabel("Bit Error Probability")
			plt.ylabel("Rx Decode Error Probability")
			plt.title("Rx Decode Error Probability for Given Individual Bit Error Probability, BSC")
			plt.yscale('log')
			plt.legend()
		\end{lstlisting}
	
		Below is the plot:\\
		\includegraphics[scale=0.35]{bsc_theor_probs}
		
		Also, here is a plot combining the simulation results from (a) with the theoretical results:
		
		\includegraphics[scale=0.35]{bsc_both_probs}
		
		
		
		\item What kind of channel would be a BSC with $ p = 0.5 $ ?  What change, if any, would you employ in your system design if you knew that $ 0.5 < p < 1 $ ?
		
		Below is a plot showing what happens as $p$ approaches (and passes) 0.5:
		
		\includegraphics[scale=0.35]{bsc_theor_probs_largep}
		
		This shows that for $p=0.5$, it doesn't matter how much repetition coding is used - the resulting channel makes the receiver generate a random value from the set $[0, 1]$.
		
		It can also be seen that the error rate is actually greater than 0.5 with $p > 0.5$. As a result, at the receiver, it would be beneficial to take the opposite of the majority decoding result - "minority decoding."
		
		
	\end{enumerate}


\pagebreak

    \item \textit{Naive  Bayes  Classifier}. On April 15, 1912,  the largest passenger liner ever made  collided  with  an  iceberg  during  her  maiden  voyage.   When  the  Titanic  sank it  killed  1502  out  of  2224  passengers  and  crew.   This  sensational  tragedy  shocked the  international  community  and  led  to  better  safety  regulations  for  ships.   One  of the reasons that the shipwreck resulted in such a loss of life was that there were not enough lifeboats for the passengers and crew.  Although there was some element of luck involved in surviving the sinking, some groups of people were more likely to survive than others. 
    
    Some demographic data are collected for 887 passengers and are provided in \textit{titanic.csv}.The price class, gender (1 for Male and 0 for Female) and age are recorded for each survived (1) or killed (0) passenger.  We use random variables \textit{S}, \textit{C}, \textit{G}, and \textit{A} for survival status, price class, gender and age, respectively.  In this problem, we are going to build a popular Naive Bayes Classifier to predict \textit{S} given \textit{C}, \textit{G}, and \textit{A}.
    
    	\begin{enumerate}
    		\item Estimate the PMFs for \textit{S}, \textit{C}, \textit{G}, and \textit{A} by finding the fraction of each realization of these random variables among all data. Plot these PMFs.
    		
    		Below is the code used to generate these PMFs:
    		
    		\begin{lstlisting}
    			import numpy as np
    			import pandas as pd
    			import matplotlib.pyplot as plt
    			
    			import seaborn as sns
    			sns.set(color_codes=True)
    			# settings for seaborn plot sizes
    			sns.set(rc={'figure.figsize':(24,10)})
    			
    			filename = 'titanic.csv'
    			rv_names = ['S', 'C', 'G', 'A']
    			
    			df = pd.read_csv('titanic.csv')
    			
    			for i in range(0, df.shape[1]):
    				col = df.iloc[:, i]
    				pmf = col.value_counts().sort_index() / len(col)
	    			fig = plt.figure()
    				#fig.suptitle("PMF of {}".format(rv_names[i]), fontsize=24)
    				ax = fig.add_subplot(111)
    				ax.set_title("PMF of {}".format(rv_names[i]), fontdict={'fontsize': 22, 'fontweight': 'medium'})
    				pmf.plot(kind="bar")
    		\end{lstlisting}
    	
    		Below are the plots generated by the above code:
    		
    		\includegraphics[scale=0.35]{pmf_s}
    		\includegraphics[scale=0.35]{pmf_c}
    		\includegraphics[scale=0.35]{pmf_g}
    		\includegraphics[scale=0.35]{pmf_a}
    		
    		
    		
    		\item We are interested in how \textit{C}, \textit{G}, and \textit{A} affect \textit{S}, in the context of the \textit{Naive  Bayes  Classifier}; the first step is to estimate the conditional PMF conditioning on the outcome of interest, i.e., survival or not.  Estimate and plot the conditional PMFs for \textit{C}, \textit{G}, and \textit{A} conditioned on \textit{S}.
    		
    		In order to do this part, I generated two separate dataframes for survival and death:
    		
    		\begin{lstlisting}
    			df_s1 = df[df["Survived"] == 1]
    			df_s0 = df[df["Survived"] == 0]
    		\end{lstlisting}
    	
    		Once I had done that, I simply generated a plot for each of the 6 conditional PMFs, similar to the process in (a):
    		
    		\begin{lstlisting}
    			for ix, col_name in enumerate(["Pclass", "Sex", "Age"]):
    				
    				#get the individual data
	    			col = df_s1[col_name]
	    			col2 = df_s0[col_name]
	    			
	    			#get the conditonal PMFs for RV
	    			pmf_cond = col.value_counts().sort_index() / len(col)
	    			pmf_cond2 = col2.value_counts().sort_index() / len(col2)
	    			
	    			#figure for PMF of RV conditioned on survival
	    			fig = plt.figure()
	    			ax = fig.add_subplot(111)    
	    			ax.set_title("PMF of {} Conditioned on Survival".format(rv_names[ix+1]), fontdict={'fontsize': 22, 'fontweight': 'medium'})
	    			pmf_cond.plot(kind="bar")
	    			
	    			#figure for PMF of RV conditioned on death
	    			fig2 = plt.figure()
	    			ax2 = fig2.add_subplot(111)
	    			ax2.set_title("PMF of {} Conditioned on Death".format(rv_names[ix+1]), fontdict={'fontsize': 22, 'fontweight': 'medium'})
	    			pmf_cond2.plot(kind="bar")
    		\end{lstlisting}
    	
    		Below are the plots generated by this notebook cell:
    		
    		\includegraphics[scale=0.35]{pmf_c_s}
    		\includegraphics[scale=0.35]{pmf_g_s}
    		\includegraphics[scale=0.35]{pmf_a_s}
    		\includegraphics[scale=0.35]{pmf_c_d}
    		\includegraphics[scale=0.35]{pmf_g_d}
    		\includegraphics[scale=0.35]{pmf_a_d}
    		
    		\pagebreak
    		
    		\item In the \textit{Naive  Bayes  Classifier}, we use the conditional independence assumption. For example, if \textit{C}, \textit{G}, and \textit{A} are conditionally independent on \textit{S}, then
    		\[
    		P(C,G,A|S=0) = (C|S=  0)P(G|S=  0)P(A|S=  0) \; and \; P(C,G,A|S=1) = (C|S=1)P(G|S=1)P(A|S=1) 
    		\]
    		
    		Using this assumption, compute
    		\[
    		P(S= 0,C= 1,G= 0,A\leq40) \; and \; P(S=1,C= 1,G= 0,A\leq40)
    		\]
    		
    		If we treat $C= 1,G= 0,A\leq40$ as one event and $S=0$ or $S=1$ as another event, we can use the definition of conditional probability to write, for example: 
    		\begin{equation}
				P(S=0,C=1,G=0,A\leq40) = P(S=0)P(C=1,G=0,A\leq40|S=0)
    		\end{equation}
    		
    		Then, taking the right side of equation (1) and using the the conditional independence assumption, we get the following relationships:
       		\begin{multline}
    			P(S=0,C=1,G=0,A\leq40) = P(S=0)P(C=1,G=0,A\leq40|S=0) \\
    			= P(S=0)[P(C=1|S=0)P(G=0|S=0)P(A\leq40|S=0)]
    		\end{multline}
    		\begin{multline}
    			P(S=1,C=1,G=0,A\leq40) = P(S=1)P(C=1,G=0,A\leq40|S=1) \\
    			= P(S=1)[P(C=1|S=1)P(G=0|S=1)P(A\leq40|S=1)]
    		\end{multline}
    	
    		With these expressions, the PMFs of $S$ computed in (a), and the conditional PMFs of the other RVs computed in (b), we can implement the calculations in Python below:
    		
    		\begin{lstlisting}
    			pmf_survival = df["Survived"].value_counts().sort_index() / len(df["Survived"])
    			
    			pS0 = pmf_survival[0]
    			pS1 = pmf_survival[1]
    			
    			pmf_s1_class = df_s1["Pclass"].value_counts().sort_index() / len(df_s1["Pclass"])
    			pmf_s0_class = df_s0["Pclass"].value_counts().sort_index() / len(df_s0["Pclass"])
    			
    			pC1S1 = pmf_s1_class[1]
    			pC1S0 = pmf_s0_class[1]
    			
    			pmf_s1_gender = df_s1["Sex"].value_counts().sort_index() / len(df_s1["Sex"])
    			pmf_s0_gender = df_s0["Sex"].value_counts().sort_index() / len(df_s0["Sex"])
    			
    			pG0S1 = pmf_s1_gender[0]
    			pG0S0 = pmf_s0_gender[0]
    			
    			pmf_s1_age = df_s1["Age"].value_counts().sort_index() / len(df_s1["Age"])
    			pmf_s0_age = df_s0["Age"].value_counts().sort_index() / len(df_s0["Age"])
    			
    			pA40S1 = pmf_s1_age.loc[pmf_s1_age.index <= 40.0].sum()
    			pA40S0 = pmf_s0_age.loc[pmf_s0_age.index <= 40.0].sum()
    			
    			#P(C, G, A|S = 0) = P(C|S = 0)P(G|S = 0)P(A|S = 0)
    			pC1G0A40gS0 = pC1S0*pG0S0*pA40S0
    			pS0C1G0A40 = pS0*pC1G0A40gS0
    			print(pS0C1G0A40)
    			print(pC1G0A40gS0)
    			
    			#P(C, G, A|S = 1) = P(C|S = 1)P(G|S = 1)P(A|S = 1)
    			pC1G0A40gS1 = pC1S1*pG0S1*pA40S1
    			pS1C1G0A40 = pS1*pC1G0A40gS1
    			print(pS1C1G0A40)
    			print(pC1G0A40gS1)
    			
    		\end{lstlisting} 
    	
    		Following the above, we obtain the following:\\
    		
    		\[
    		P(S=0,C=1,G=0,A\leq40) = 0.010625322687488965
    		\]
    		
    		\[
    		P(S=1,C=1,G=0,A\leq40) = 0.08460553314142814
    		\]\\
    	 		
    		\item Based on your result in (c), compute 
    		\[
    		P(S= 0|C= 1,G= 0,A\leq40) \; and \; P(S=1|C= 1,G= 0,A\leq40)
    		\]
    		Predict whether a female whose age is under 40 and who is in first class will survive or not.
    		
    		
    		Using Bayes's Rule, we can obtain an expression for the above:
    		
    		\begin{equation}
    			P(S= 0|C= 1,G= 0,A\leq40) = \frac{P(S=0)P(C=1,G=0,A\leq40|S=0)}{P(C=1, G=0,A \leq 40)}
    		\end{equation}

    		\begin{equation}
    			P(S=1|C= 1,G= 0,A\leq40) = \frac{P(S=1)P(C=1,G=0,A\leq40|S=1)}{P(C=1, G=0,A \leq 40)}
    		\end{equation}
    	
    		And then recognizing that the right side of each of these expressions is just the corresponding result from (c), we can make the substitution and obtain:
    		
    		\begin{equation}
    			P(S= 0|C= 1,G= 0,A\leq40) = \frac{P(S=0,C=1,G=0,A\leq40)}{P(C=1, G=0,A \leq 40)}
    		\end{equation}
    		
    		\begin{equation}
    			P(S=1|C= 1,G= 0,A\leq40) = \frac{P(S=1,C=1,G=0,A\leq40)}{P(C=1, G=0,A \leq 40)}
    		\end{equation}
    		
    		Now, all we need is a way to compute $P(C=1, G=0,A \leq 40)$. For this, we can use the Law of Total Probability, recognizing that $S=0$ and $S=1$ form a partition over $S$, and simply sum the two results from (c):
    		\begin{equation}
    			P(C=1, G=0,A \leq 40) = P(S=0,C=1,G=0,A\leq40) + P(S=1,C=1,G=0,A\leq40)
    		\end{equation}
    	
    		Substituting (8) into (6) and (7), we now have an expression for the desired result in terms of known quantities. Below is the code I used to compute these quantities:
    		
    		\begin{lstlisting}
    			df_first_class = df[df["Pclass"]==1]
    			df_first_class_female = df_first_class[df_first_class["Sex"]==0]
    			df_first_class_female_u40 = df_first_class_female[df_first_class_female["Age"]<=40]
    			
    			#Use total probability law to estimate P(C=1,G=0,A<=40)
    			pC1G0A40 = pS0C1G0A40 + pS1C1G0A40
    			
    			pS1gC1G0A40 = pS1C1G0A40/pC1G0A40
    			print("Estimate for survival given person is a first class woman under 40 on titanic: {}".format(pS1gC1G0A40))
    			
    			pS0gC1G0A40 = pS0C1G0A40/pC1G0A40
    			print("Estimate for death given person is a first class woman under 40 on titanic: {}".format(pS0gC1G0A40))
    		\end{lstlisting}
    	
    		The resulting values for $P(S= 0|C= 1,G= 0,A\leq40)$ and $P(S=1|C= 1,G= 0,A\leq40)$, respectively, were 0.112 and 0.888. As a result, it was more likely for a female whose age is under 40 and who is in first class to survive than to die aboard the Titanic.
    		
    	\end{enumerate}


	\pagebreak
	
	\item \textit{Central  Limit  Theorem}. Let $X_1, X_2, X_3, \ldots$  be  a  sequence  of  i.i.d.   random variables with finite mean $\mu$ and finite variance $\sigma^2$, and let $Z_n$ be the sum of the first $ n $ random variables in the sequence:
	
	\[
	Z_n=X_1+X_2+\ldots+X_n.
	\]
		
		\begin{enumerate}
			\item Let $X_n$ for $i=1,2,\ldots$ be a uniform continuous random variable taking values in the interval (1,4). Write a MATLAB program to plot the pdf of $Z_n$. Consider $n=1,3,10,30$ and compare your results across different $n$'s.
			
			The below python notebook cell will plot separate PDFs for each value of $n$:
			
			\begin{lstlisting}
				import numpy as np
				import pandas as pd
				import matplotlib.pyplot as plt
				
				import seaborn as sns
				sns.set(color_codes=True)
				# settings for seaborn plot sizes
				sns.set(rc={'figure.figsize':(10,10)})
				
				n_vec = [1, 3, 10, 30]
				n_trials = 100000
				
				colors = ['blue', 'green', 'red', 'orange']
				
				def get_xi():
					return np.random.uniform(1, 4)
				
				for n in n_vec:
					sums = []
				
					for i in range(0, n_trials):
				
						values = []
				
						for j in range(0, n):
							value = get_xi()
							values.append(value)
				
						sums.append(np.sum(values)/n)
				
					plt.figure()
				
					ax = sns.distplot(sums,
									  bins=100,
									  label="n={}".format(n),
									  kde=False,
									  color=colors[n_vec.index(n)],
									  hist_kws={"linewidth": 1,'alpha':0.8})
									  ax.set(xlabel='Value of Sum of {} Uniform RVs in {} Trials'.format(n, n_trials), ylabel='Frequency')
				
				
				
			\end{lstlisting}
			
			I'm not going to include the individual plots in the interest of brevity. Below is the code to generate a comparison plot of the PDFs of $Z_n$ for different values of $ n $:\\
			
			\begin{lstlisting}
				sums_list = []
				
				for n in n_vec:
					sums = []
				
					for i in range(0, n_trials):
				
						values = []
				
						for j in range(0, n):
							value = get_xi()
							values.append(value)
				
						sums.append(np.sum(values)/n)
				
				
					ax = sns.distplot(sums,
									  bins=100,
									  label="n={}".format(n),
								   	  kde=False,
									  color=colors[n_vec.index(n)],
									  hist_kws={"linewidth": 1,'alpha':0.8})
									  ax.set(xlabel='Value of Sum of {} Uniform RVs in {} Trials'.format(n, n_trials), ylabel='Frequency')
			\end{lstlisting}
			
			Below is the plot:\\
			
			\includegraphics[scale=0.5]{Zn_pdf_comparison}
			
			\item Calculate analytically the mean and the variance of $X_i$ and $Z_n$ in part (a).
			
			\[
			E[Z_n] = E[X_1 + X_2 + \ldots + X_n]
			=E[X_1] + E[X_2] + \ldots + E[X_n]
			= n\mu
			\]
			
			\pagebreak
			\begin{equation}
				\begin{split}
					VAR[Z_n] = E[(Z_n - E[Z_n])^2] \\
					& = E[((X_1 + X_2 + \ldots + X_n) - n\mu)^2] \\   
					& = E[(X_1 - \mu)^2 + (X_2 - \mu)^2 + \ldots + (X_n - \mu)^2] \\
					& = VAR[X_1] + VAR[X_2] + \ldots + VAR[X_n] \\
					& = n\sigma^2
				\end{split}
			\end{equation}
		
			Where, in (9), the result is supported by Bienayme's formula.
			
			\item Write a MATLAB program to generate a Gaussian random variable with the same mean and variance as $Z_n$. Superimpose its pdf on the plots from part (a).\\
			
			The mean of a continuous uniform RV in the range [1, 4] is 2.5 and the variance is $\frac{3}{4}$. Following the results in (b), we can implement some code to plot a Gaussian with mean $2.5n$ and standard deviation $0.75n$, as below:
			
			\begin{lstlisting}
				## Part c: superimposing a gaussian:
				
				n_vec = [1, 3, 10, 30]
				n_trials = 100000
				
				colors = ['blue', 'green', 'red', 'orange']
				
				def get_xi():
					return np.random.uniform(1, 4)
				
				sums_list = []
				
				
				plt.figure()
				
				for n in n_vec:
					sums = []
				
					for i in range(0, n_trials):
				
						values = []
				
						for j in range(0, n):
							value = get_xi()
							values.append(value)
				
				
						sums.append(np.sum(values))
						#Y = (np.sum(values)/np.sqrt(n)) + 2.5*(1-np.sqrt(n))
						#sums.append(Y)
				
					sums_list.append(sums)
				
					ax = sns.distplot(sums,
									bins=int(20*np.sqrt(n)),
									hist=True,
									label="n={}".format(n),
									kde=False,
									color=colors[n_vec.index(n)],
									hist_kws={"linewidth": 0.25,'alpha': 0.8, "density": True})
					ax.set(xlabel='Value of Sum of N Uniform RVs over {}k Trials'.format(int(n_trials/1000)), ylabel='Frequency')
					ax.legend()
				
					#generate the gaussian
					mu, sigma = 2.5*n, 0.75*np.sqrt(n) # mean and standard deviation
					s = np.random.normal(mu, sigma, 1000)
				
					hist, bins = np.histogram(s, 30, density=True)
					plt.plot(bins, 1/(sigma * np.sqrt(2 * np.pi)) * np.exp( - (bins - mu)**2 / (2 * sigma**2) ), linewidth=2, color='r')
			\end{lstlisting}
		
			And below is the plot:\\
			
			\includegraphics[scale=0.5]{Zn_pdf_with_gaussian}
			
			\item Repeat parts (a), (b), and (c) with $X_i$ representing a toss of a fair 4-sided die (see Problem 1). Note that $X_i$ and $Z_n$ are discrete in this case.
			
			The Python notebook cell containing the code to do this is reproduced below:
			
			\begin{lstlisting}
				n_vec = [1, 3, 10, 30]
				n_trials = 100000
				
				die = np.arange(1,5)
				probs_unbiased = [(1/4)]*4
				
				colors = ['blue', 'green', 'red', 'orange']
				
				def get_xi():
					return np.random.choice(die, p=probs_unbiased) #throw the fair die t times
				
				
				plt.figure()
				for n in n_vec:
					sums = []
				
					for i in range(0, n_trials):
				
						values = []
				
						for j in range(0, n):
							value = get_xi()
							values.append(value)
				
				
						sums.append(np.sum(values))
				
					sums_list.append(sums)
				
					ax = sns.distplot(sums,
							bins=int(20*np.sqrt(n)),
							hist=True,
							label="n={}".format(n),
							kde=False,
							color=colors[n_vec.index(n)],
							hist_kws={"linewidth": 0.25,'alpha': 0.8, "density": True})
					ax.set(xlabel='Value of Sum of N Uniform RVs over {}k Trials'.format(int(n_trials/1000)), ylabel='Frequency')
					ax.legend()
				
					#generate the gaussian
					mu, sigma = 2.5*n, 0.75*np.sqrt(n) # mean and standard deviation
					s = np.random.normal(mu, sigma, 1000)
				
					hist, bins = np.histogram(s, 30, density=True)
					plt.plot(bins, 1/(sigma * np.sqrt(2 * np.pi)) * np.exp( - (bins - mu)**2 / (2 * sigma**2) ), linewidth=2, color='r')
			\end{lstlisting}
		
			And the generated plot appears below:
			
			\includegraphics[scale=0.5]{Zn_discr_pdf_with_gaussian}
			
			For the repetition of part (b), nothing really changes due to the fact that the $X_i$s are discrete rather than continuous. That is a commonly stated property of CLT.

			
		\end{enumerate}

\end{enumerate}









\end{document}
